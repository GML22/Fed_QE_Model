%****************************ZAKOŃCZENIE*******************************
\newpage
\chapter*{Zakończenie}
 \addcontentsline{toc}{chapter}{Zakończenie}
 
Niniejsza praca podejmowała problematykę niekonwencjonalnej polityki monetarnej stosowanej przez banki centralne w~sytuacji, gdy konwencjonalne narzędzia przestały przynosić spodziewane efekty. Praca ta skupiała się na zdefiniowaniu najważniejszych założeń \acs{NPM}, opisie zastosowania jej podstawowych instrumentów oraz kanałów jej transmisji. Celem pracy było zbadanie skuteczności tego rodzaju polityki monetarnej w~pobudzaniu wzrostu gospodarczego na przykładzie Stanów Zjednoczonych. Przyjęta na początku pracy hipoteza badawcza zakładała, iż niekonwencjonalna polityka monetarna Rezerwy Federalnej stosowana od 2008~roku zamiast pobudzać do wzrostu realnego \acs{PKB} oraz~zbliżać amerykańską gospodarkę do pełnego zatrudnienia przyczyniła się przede wszystkim do wygenerowania ponadprzeciętnych wzrostów cen akcji notowanych na giełdach w~Stanach Zjednoczonych.

Pierwszy rozdział przedstawiał podstawy teoretyczne niekonwencjonalnej polityki monetarnej od jej najważniejszych założeń, przez jej najpopularniejsze narzędzia na kanałach jej transmisji do realnej gospodarki kończąc. Celem tego rozdziału było zbudowanie fundamentów teoretycznych umożliwiających analizę dalszych części pracy. Drugi rozdział stanowił przegląd wybranych badań z~zakresu niekonwencjonalnej polityki monetarnej prowadzonej przez Rezerwę Federalną w~latach 2008-2016. Wnioski uzyskane w~tym rozdziale z~analizy literatury badawczej posłużyły jako punkt wyjścia do właściwego zdefiniowania modelu badawczego oraz stanowiły punkt odniesienia w~stosunku do dalszych analiz. Rozdział trzeci poświęcony został zdefiniowaniu modeli badawczych oraz analizie wpływu szoków niekonwencjonalnej polityki pieniężnej na kształt krzywej dochodowości amerykańskich obligacji skarbowych oraz na pozostałe wskaźniki gospodarcze. W~rozdziale tym udało się wykazać statystycznie istotny wpływ szoku \acs{NPM} w~skali 120 miliardów dolarów na: wzrost rentowności długoterminowych obligacji skarbowych Stanów Zjednoczonych o~około 45~punktów bazowych oraz wzrost indeksu S\&P500 o~ponad 150~punktów indeksowych.

Hipoteza badawcza zawarta we wstępie niniejszej pracy została potwierdzona jedynie połowicznie - udało się wykazać statystycznie istotny wpływ niekonwencjonalnej polityki monetarnej na wygenerowanie ponadprzeciętnych wzrostów cen akcji notowanych na giełdach w~Stanach Zjednoczonych. Drugiej części hipotezy badawczej nie udało się ani potwierdzić, ani odrzucić - wpływ nadzwyczajnych działań amerykańskich władz monetarnych po 2008~roku na gospodarkę Stanów Zjednoczonych okazał się niejednoznaczny. Cześć wskaźników gospodarczych wykazała drobne reakcje na szoki niekonwencjonalnej polityki monetarnej - skala tych reakcji nie była znacząca z~wyjątkiem PKB realnego, gdzie reakcją na szok \acs{NPM} był wzrost o~50 punktów bazowych, jednak dalsze analizy pokazały, iż reakcja ta może być obarczona błędem słabego dopasowania modelu.

Zagadnienie niekonwencjonalnej polityki pieniężnej banków centralnych i~jej wpływu na gospodarkę oraz rynki finansowe, co oczywiste, nie zostało w~niniejszej pracy wyczerpane. Otwarta pozostaje kwestia wpływu \acs{NPM} na amerykańską gospodarkę, co daje pole do dalszych prac badawczych, w~sytuacji gdy dostępne będą dane z~dłuższego horyzontu badawczego, zawierające w~szczególności okres wychodzenie amerykańskiej gospodarki z~niekonwencjonalnej polityki monetarnej. Warty dokładniejszego zbadania jest też wpływ \acs{NPM} na krzywą dochodowości amerykańskich obligacji skarbowych, gdyż wyniki uzyskane w~niniejszej pracy w~tej kwestii wydają się stać w~sprzeczności z~wynikami pojawiającymi się w~dużej części literatury badawczej. Mimo to wydaje się, iż uprawnione jest stwierdzenie, że zaprezentowane w~niniejszej pracy badanie wnosi nowe, istotne obserwacje do literatury dotyczącej niekonwencjonalnej polityki monetarnej.
%*********************************Streszczenie itp****************************************

\newpage

\vspace{1cm}
\section*{Streszczenie}

\noindent Niniejsza Praca podejmuje problematykę skuteczności niekonwencjonalnej polityki monetarnej stosowanej przez Rezerwę Federalną w~latach 2008-2016, w~szczególności siły jej oddziaływania na amerykańską gospodarkę oraz rynki finansowe. W~kolejnych częściach opisywane są podstawy teoretyczne niekonwencjonalnej polityki monetarnej, literatura badawcza dotycząca problematyki oraz badanie statystyczne mające na celu zweryfikowanie najważniejszych hipotez. Głównym narzędziem badawczym jest model VAR oszacowany na danych miesięcznych pochodzących z~amerykańskiej gospodarki dla lat 2008-2016. Najważniejsze wnioskami są: stwierdzenie istotnego wpływu niekonwencjonalnej polityki monetarnej na wzrost indeksów na giełdach w~USA oraz na wzrost rentowności amerykańskich obligacji skarbowych.

\vspace{1cm}
\section*{Słowa kluczowe}

\noindent niekonwencjonalna polityka monetarna, kryzys gospodarczy, luzowanie pieniężne, Rezerwa Federalna

\vspace{1cm}
\section*{Dziedzina pracy (kody wg programu Socrates-Erasmus)}

\noindent Ekonomia (14300)

\vspace{1cm}
\section*{Tytuł pracy w języku angielskim}

\noindent The effectiveness of unconventional monetary policy of the Federal Reserve during the economic crisis. Analysis for the period of 2008-2016.

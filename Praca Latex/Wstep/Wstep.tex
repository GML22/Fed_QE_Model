%*********************************WSTĘP****************************************
\newpage
\chapter*{Wstęp}
 \addcontentsline{toc}{chapter}{Wstęp}
 
Wybuch kryzysu gospodarczego w~2008~roku zaburzył równowagę gospodarczą większości krajów współczesnego świata stawiając im nowe wyzwania w~zakresie zarówno polityki fiskalnej, jak i~monetarnej. Szczególnie mocno naruszył on fundamenty gospodarcze takich krajów rozwiniętych jak: Japonia, Stany Zjednoczone, Wielka Brytania, czy strefa euro. Władze monetarne tych krajów po wykorzystaniu większości konwencjonalnych narzędzi polityki pieniężnej zdecydowały się na zastosowanie nadzwyczajnych działań, które miały na celu pobudzenie wzrostu gospodarczego poprzez stymulowanie akcji kredytowej banków komercyjnych. Zespół tych nadzwyczajnych działań władz monetarnych został następnie określony mianem niekonwencjonalnej polityki monetarnej (ang. \textit{Unconventional Monetary Policy - UMP})\footnote{W dalszej części pracy określana skrótem \acs{NPM}} po to by zaznaczyć ich odrębność w~stosunku do konwencjonalny metod działań banków centralnych. Poniższa praca powstała w~celu zgłębienia tematyki niekonwencjonalnej polityki monetarnej, starając się zarysować jej podstawy teoretyczne oraz sprawdzić jej skuteczność w~kontekście oddziaływania na gospodarkę oraz rynki finansowe.

W niniejszej pracy autor jako hipotezę badawczą przyjął, iż \textbf{niekonwencjonalna polityka monetarna Rezerwy Federalnej stosowana od 2008~roku zamiast pobudzać do wzrostu realnego \acs{PKB} oraz~zbliżać amerykańską gospodarkę do pełnego zatrudnienia przyczyniła się przede wszystkim do wygenerowania ponadprzeciętnych wzrostów cen akcji notowanych na giełdach w~Stanach Zjednoczonych}. Kolejne rozdziały niniejszej pracy maję na celu zebranie informacji koniecznych do zweryfikowania tej hipotezy.

Pierwszy rozdział poświęcony został zarysowaniu podstaw teoretycznych niekonwencjonalnej polityki monetarnej, w~tym w~szczególności zdefiniowaniu i~usystematyzowaniu najważniejszych jej instrumentów oraz kanałów jej transmisji do realnej gospodarki. Wydaje się to niezbędne by móc bez wątpliwości przystąpić do weryfikacji hipotezy badawczej. Drugi rozdział stanowi przegląd wybranych artykułów badawczych poruszających zagadnienie \acs{NPM} w~kontekście jej zastosowania w~przez Rezerwę Federalnej w~latach 2008-2016. Analiza przeprowadzona w~tym rozdziale pozwoli przeanalizować hipotezy badawcze zawarte w~innych pracach badawczych oraz sposoby ich weryfikacji. Ostatni rozdział początkowo koncentruje się na opracowaniu odpowiednich modeli statystycznych, które posłużą do weryfikacji hipotezy badawczej, następnie dochodzi do ich oszacowania, przedstawienia wyników, wyciągnięcia finalnych wniosków  oraz weryfikacji hipotez.

Jako główne narzędzie badawcze w~niniejszej pracy posłuży model wektorowej autoregresji (ang. \textit{Vector Autoregressive Model - VAR}), który został wybrany ze względu na dużą częstość stosowania w~literaturze badawczej poruszającej zagadnienie niekonwencjonalnej polityki monetarnej. Badanie zostanie przeprowadzone dla lat 2008-2016, na danych miesięcznych z~amerykańskiej gospodarki, w~ogromnej większości pochodzących ze strony internetowej Banku Rezerwy Federalnej Saint Louis (\url{http://research.stlouisfed.org}). Wyniki uzyskane z~modelu \acs{VAR} zostaną przeanalizowane za pomocą: funkcji odpowiedzi na impuls, dekompozycji wariancji błędów prognoz oraz analizy wstecznej trafności prognoz uzyskanych z~modelu.
\newpage
\chapter*{Rozdział 2 \\ \vspace{1cm} \Large{Przegląd wybranych badań z~zakresu niekonwencjonalnej polityki monetarnej}}
 \addcontentsline{toc}{chapter}{2. Przegląd wybranych badań z~zakresu niekonwencjonalnej polityki monetarnej}
 
W~niniejszym rozdziale zaprezentowane zostaną wybrane badania naukowe z~zakresu niekonwencjonalnej polityki monetarnej stosowanej przez Rezerwę Federalną po 2008~roku. Przegląd literatury badawczej będzie się skupiać przede wszystkim na analizie artykułów poruszających kwestię efektywności wpływu niekonwencjonalnej polityki monetarnej na amerykańską gospodarkę. Dzięki temu analiza przedstawiona w~bieżącym rozdziale będzie mogła stanowić punkt odniesienia do przedstawionego w~rozdziale trzecim badania własnego.

Pierwszym analizowanym badaniem będzie artykuł \textbf{\textit{The Financial Market Effects of the Federal Reserve’s Large-Scale Asset Purchases}}, którego autorami są J. Gagnon (Instytut Petersona ds. Ekonomii Międzynarodowej), M. Raskin, J.Remache oraz B. Sack (wszyscy  z Banku Rezerwy Federalnej w~Nowym Jorku). Został on opublikowany w~marcu 2011~roku w~\textit{International Journal of Central Banking} \cite{gagnon34}. Celem autorów artykułu była analiza skuteczności niekonwencjonalnej polityki monetarnej Rezerwy Federalnej w~pierwszych latach po wybuchu światowego kryzysu gospodarczego z~2008~roku. Szczegółowej analizie zostały poddane reakcje rentowności długoterminowych amerykańskich obligacji skarbowych na ogłoszenie pierwszego programu luzowania ilościowego przeprowadzonego przez Rezerwę Federalną w~latach 2008-2010. 

Aby zrealizować cel badawczy autorzy analizowanego artykułu posłużyli się dwoma metodami badawczymi: \textit{event study} oraz modelem najmniejszych kwadratów (ang. \textit{Ordinary Least Squares - OLS}) w~jego normalnej oraz dynamicznej formie (z~opóźnieniami zmiennych zależnych). W~badaniu \textit{event study} zbadano reakcję premii za czas do wykupu długoterminowych amerykańskich papierów dłużnych na ogłaszanie kolejnych komunikatów związanych z~niekonwencjonalną polityką monetarną. W~modelu najmniejszych kwadratów zmienną zależną była premia za czas do wykupu dziesięcioletnich amerykańskich papierów wartościowych, a~zmiennymi niezależnymi: luka bezrobocia (różnica pomiędzy bieżącą stopą bezrobocia a~naturalną stopą bezrobocia), inflacja bazowa \acs{CPI}, długoterminowa niezgodność inflacji (mierzona jako międzykwartylowy rozstęp 5-cio do 10-cio letnich oczekiwań inflacyjnych), sześciomiesięczna zrealizowana dzienna zmienność dziesięcioletnich obligacji skarbowych, amerykańskie papiery skarbowe w~publicznym posiadaniu z~co najmniej jednoroczną zapadalnością (jako procent nominalnego \acs{PKB}), papiery skarbowe trzymane w~portfelu Rezerwy Federalnej z~co najmniej jednoroczną zapadalnością (jako procent nominalnego \acs{PKB}), amerykańskie obligacje skarbowe posiadane przez zagraniczne podmioty (jako procent nominalnego \acs{PKB}). Wszystkie zmienne wykorzystane w~badaniu dotyczyły amerykańskiej gospodarki, ich obserwacje miały miesięczną częstotliwość i~pochodziły z~lat 1985-2008.

Najważniejszym wnioskiem przedstawionym w~analizowanym artykule było określenie siły wpływu ogłoszenia pierwszego luzowania ilościowego przez Rezerwę Federalną na obniżenie premii za czas do wykupu dziesięcioletnich amerykańskich papierów wartościowych. Analiza \textit{event study} wskazała, iż wpływ ten wynosił od 50~do 100~punktów bazowych, natomiast model najmniejszych kwadratów wskazał, iż zakup dłużnych instrumentów finansowych o~wartości 1,725 biliona dolarów przez amerykańskie władze monetarne mógł wpłynąć na obniżenie długoterminowej premii za czas do wykupu w~skali od 38~do 82 punktów bazowych. Dodatkowymi wnioskami przestawionymi w~analizowanym artykule były: pozytywny wpływ niekonwencjonalnej polityki monetarnej na poprawę płynności na rynkach instrumentów dłużnych oraz usunięcie szkodliwych aktywów z~portfeli prywatnych inwestorów. Największą słabością analizowanego artykułu wydaje się użycie w~konstrukcji modelu najmniejszych kwadratów obserwacji z~okresu sprzed wybuchu światowego kryzysu gospodarczego do wyciągania wniosków odnośnie interakcji pomiędzy poszczególnymi zmiennymi po tym kryzysie, kiedy to zależności pomiędzy analizowanymi zmiennymi mogły ulec drastycznym zmianom. Warto więc byłoby przeprowadzić podobne badanie na wystarczająco długiej próbie obejmującej obserwacje wyłącznie z~lat po wybuchu światowego kryzysu gospodarczego.

Następnym poddanym analizie artykułem badawczym w~bieżącym rozdziale jest \textbf{\textit{The Effectiveness of Unconventional Monetary Policy at the Zero Lower Bound: A Cross-Country Analysis}} autorstwa L. Gambacorta, B. Hofmann (obaj z Banku Rozrachunków Międzynarodowych) oraz G. Peersman (Uniwersytet Gandawa). Artykuł ten został opublikowany w~czerwcu 2014~roku w~\textit{Journal of Money, Credit and Banking} \cite{gambacorta35}. Celem analizowanego artykułu było zbadanie makroekonomicznej efektywności niekonwencjonalnej polityki monetarnej w~pierwszych latach kryzysu gospodarczego w~ośmiu gospodarkach rozwiniętych, którymi były: Kanada, Japonia, Norwegia, Szwajcaria, Szwecja, Wielka Brytania, strefa euro oraz Stany Zjednoczone. Autorzy badania w~szczególności położyli nacisk na zbadanie wpływu wzrostu wartości bilansu banku centralnego na produkt krajowy brutto danej gospodarki, poziom zmienności na krajowej giełdzie oraz poziom cen. 

Jako metoda badawcza w~artykule Gambacorta, Hofmanna i~Peersmana posłużył strukturalny model wektorowej autoregresji (ang. \textit{Structural Vector Autoregression - SVAR}) , gdzie zmiennymi endogenicznymi były: logarytm naturalny wyrównanego sezonowo realnego PKB, logarytm naturalny wyrównanego sezonowo acs{CPI}, logarytm naturalny wyrównanych sezonowo aktywów banku centralnego oraz poziom implikowanej zmienności krajowego indeksu akcji. Bazując na analizie kryteriów informacyjnych autorzy analizowanego badania zdecydowali się na zastosowanie w~modelu \acs{SVAR} dwóch opóźnień zmiennych endogenicznych. Jako próbę badawczą posłużyły im miesięczne obserwacje od stycznia 2008~roku do czerwca 2011~roku, zostały one pozyskane z~baz danych Banku Rozrachunków Międzynarodowych, \textit{Datastream} oraz krajowych źródeł. Wyniki uzyskane z~modelu \acs{SVAR} zostały przeanalizowane przez autorów omawianego badania za pomocą dwóch podstawowych narzędzi: funkcji odpowiedzi na impuls oraz dekompozycji wariancji błędów prognoz. Aby zapewnić odporność wniosków wynikających ze swojego modelu na jego błędy konstrukcyjne autorzy artykułu \textit{The Effectiveness of Unconventional Monetary Policy at the Zero Lower Bound: A Cross-Country Analysis} postanowili oszacować model \acs{SVAR} w~pięciu dodatkowych wariantach, zmieniając w~każdym z~nich inne konstrukcyjne założenie.

Najważniejszym wnioskiem płynącym z~opisywanego artykułu jest stwierdzenie, iż pomimo różnic w~implementacji niekonwencjonalnej polityki monetarnej pomiędzy poszczególnymi krajami jej efekty były bardzo podobne we wszystkich badanych krajach - niekonwencjonalna polityka monetarna w~pierwszych latach kryzysu gospodarczego (tj. 2008-2011) przyczyniła się do tymczasowego wsparcie gospodarek krajów, w~których została zastosowana. Autorzy stwierdzają w~szczególności istotny wpływ niekonwencjonalnej polityki monetarnej na poziom dochodu narodowego oraz cen, które po szoku wzrostu bilansu Rezerwy Federalnej o~około 3\% reagują (w~perspektywie 24~miesięcy) wzrostem na poziomie średnio 0,05 punktu procentowego (\acs{PKB}) oraz 0,02 p.p. (\acs{CPI}). Z~kolei implikowana zmienność cen akcji (\acs{VIX}) zmniejsza się o~około 1 p.p. w~reakcji na szok, jednak efekt ten niweluje się całkowicie po sześciu miesiącach.  Gambacort, Hofmann i~Peersman konkludują, iż takie wyniki ich badań nie oznaczają jednak, iż wzrost bilansu banku centralnego jest w~każdych okolicznościach pozytywny dla gospodarki, w~której dany bank centralny kontroluje politykę monetarną, gdyż badanie dotyczyło tylko krótkiego okresu silnych wahań gospodarczych. Największą słabością analizowanego artykułu wydaje się stosunkowo mała liczebność próbki dla poszczególnych krajów - miesięczne dane z~3,5~roku światowego kryzysu przekładają się na 42~obserwacje dla każdego z~ośmiu analizowanych krajów. Wydaje się, iż aby móc wyciągać generalne wnioski dotyczące skuteczności niekonwencjonalnej polityki pieniężnej należałby posłużyć się próbką przynajmniej dwa razy dłuższą, tak aby móc uchwycić jeden pełny cykl koniunkturalny.

Kolejnym artykułem badawczym przeanalizowanym w~niniejszym rozdziale będzie \textbf{\textit{The Distributive Effects of Conventional and Unconventional Monetary Policies}}, którego autorem jest K. Davtyan z~Uniwersytetu w~Barcelonie. Artykuł ten został opublikowany w~maju 2016~roku na stronie Instytutu Badań nad Stosowaną Regionalną i~Publiczną Ekonomią w~Barcelonie \cite{davtyan35}. Głównym celem autora artykułu było zbadanie wpływu konwencjonalnej i~niekonwencjonalnej polityki monetarnej na rozkład dochodu narodowego w~amerykańskiej gospodarce, który mierzony był za pomocą współczynnika Giniego, czyli wskaźnika liczbowego wykorzystywanego do pomiaru poziom nierówności społecznych.

Modelem badawczym wykorzystanym w~omawianym artykule K. Davtyana jest strukturalny model wektorowej autoregresji z~następującymi zmiennymi endogenicznymi pochodzącymi z~amerykańskiej gospodarki: \acs{PKB} realne, deflator \acs{PKB}, efektywna stopa funduszy federalnych, wartość aktywów w~bilansie Rezerwy Federalnej, baza monetarna oraz współczynnik Giniego. Ważnym założeniem stworzonego przez Davtyana modelu były restrykcyjne szoki konwencjonalnej polityki monetarnej (ograniczanie bazy monetarnej) oraz ekspansywne szoki polityki niekonwencjonalnej (rozszerzanie bazy monetarnej). Oknem obserwacyjnym dla zastosowanego modelu były miesięczne obserwacje z~lat 1983-2013, przy czym badanie dotyczące konwencjonalnej polityki monetarnej zostało przeprowadzone dla lat 1983-2008, a~dla jej niekonwencjonalnej formy dla lat 2009-2013. Dla większości zmiennych wykorzystanych w~badaniu obserwacje w~interwale miesięcznym nie były dostępne stąd autor analizowanego artykułu posłużył się interpolacją z~danych kwartalnych oraz rocznych (współczynnik Giniego). Wszystkie dane wykorzystane w~badaniu zostały pozyskane z~baz danych OECD oraz Banku Rezerwy Federalnej w~Saint Louis. Podobnie jak w~przypadku badania Gambacorta, Hofmanna i~Peersmana tak i~w~omawianym artykule wyniki modelu \acs{SVAR} były interpretowane za pomocą analizy funkcji odpowiedzi na impuls oraz dekompozycji wariancji błędów prognoz.

Wyniki analizowanego badania wskazują, iż szok konwencjonalnej restrykcyjnej polityki monetarnej Rezerwy Federalnej mógł wpłynąć w~perspektywie 20~miesięcy na: spadek realnego \acs{PKB} o~około 0,35\%, spadek poziomu cen o~około 0,25\% oraz obniżenie współczynnika Giniego o~około 0,1~punktu procentowego (spadek nierówności społecznych). Natomiast szok ekspansywnej niekonwencjonalnej polityki monetarnej mógł doprowadzić do następujących zmian gospodarczych: wzrostu realnego \acs{PKB} o~około 0,25\%, wzrostu poziomu cen o~blisko 0,15\% oraz wzrostu nierówności społecznych, mierzonego za pomocą współczynnika Giniego, o~około 0,07 punktu procentowego. Analiza dekompozycji wariancji błędów prognoz wskazała natomiast, iż szoki niekonwencjonalnej polityki monetarnej tłumaczą zdecydowanie większą część zmienności współczynnika Giniego (40,71\%) niż szoki jej konwencjonalnej odmiany (11,48\%). Dużą słabością analizowanego artykułu jest zastosowanie interpolacji, w~celu uzyskania danych miesięcznych, dla większości zmiennych wykorzystywanych w~badaniu. Jest to szczególnie problematyczne dla modelu badającego wpływ niekonwencjonalnej polityki monetarnej, gdyż opierał się on na danych z~zaledwie 4~lat co oznacza, iż przy założeniu posiadania danych kwartalnych wykorzystywał on zaledwie 16 rzeczywistych obserwacji i~32~obserwacje otrzymane przy pomocy interpolacji, co może poddawać w~wątpliwość uzyskane przez autora badania wyniki. Podobnie jak w~artykule Gambacorta, Hofmanna i~Peersmana wydaje się, iż nieinterpolowana próbka danych miesięcznych dla badania skuteczności niekonwencjonalnej polityki monetarnej powinna obejmować dłuższy okres niż cztery lata aby móc uchwycić przynajmniej jeden pełny cykl koniunkturalny.

Następnym artykułem, który zostanie omówiony w~niniejszym rozdziale będzie badanie Q. Chen, A. Filardo, D. He oraz F. Zhu (Banku Rozrachunków Międzynarodowych) o~tytule \textbf{\textit{Financial crisis, US unconventional monetary policy and international spillovers}}. Badanie to zostało opublikowane w~październiku 2016~roku w~\textit{Journal of International Money and Finance} \cite{chen36}. Celem autorów badania było określenie wpływu niekonwencjonalnej polityki monetarnej Rezerwy Federalnej nie tylko na gospodarkę Stanów Zjednoczonych, ale też na gospodarkę innych wybranych krajów świata. Szczególny nacisk został położony na zbadanie wpływu szoków w~wartościach spreadów amerykańskich obligacji skarbowych oraz obligacji korporacyjnych na gospodarki krajów rozwiniętych (Stany Zjednoczone, strefa euro, Japonia, Zjednoczone Królestwo) oraz rozwijających się (Chiny, Hong Kong, Indie, Indonezja, Korea Południowa, Malezja, Filipiny, Singapur, Tajlandia, Argentyna, Brazylia, Chile, Meksyk).

Metodą badawczą wykorzystaną przez autorów omawianego artykułu był model wektorowej korekty błędem (ang. \textit{Vector Error Correction Model - VECM}), zbudowany oddzielnie dla każdego z~analizowanych krajów rozwiniętych i~rozwijających się. Każdy z~krajowych modeli składał się z~następujących zmiennych endogenicznych dotyczących gospodarki badanego kraju: \acs{PKB} realnego, wskaźnika \acs{CPI}, indeksu \acs{VIX}, współczynnika inflacji cen akcji na krajowej giełdzie, wzrostu wartości kredytów udzielonych sektorowi prywatnemu oraz presji kursu walutowego. Poza wymienionymi zmiennymi dodatkowa zmienna reprezentowała politykę monetarną Rezerwy Federalnej i~występowała w~dwóch wariantach: spreadu pomiędzy rentownościami 10-letnich amerykańskich obligacji skarbowych a~rentownościami 3-miesięcznych bonów skarbowych Stanów Zjednoczonych (nazywanego w~artykule spreadem za czas do wykupu) oraz spreadu pomiędzy rentownościami amerykańskich obligacji korporacyjnych z~ratingiem AAA\footnote{Bank of America Merrill Lynch US corporate AAA bond yield} a~efektywną stopą funduszy federalnych (nazywanego w~artykule spreadem korporacyjnym). Jako okres badawczy dla wszystkich oszacowanych modeli \acs{VECM} przyjęto miesięczne dane od lipca 2007~do lutego 2013. Wyniki oszacowanych modeli wektorowej korekty błędem były analizowane przez autorów artykułu za pomocą funkcji odpowiedzi na impuls. 

Najważniejszym wnioskiem płynącym z~analizy artykułu \textit{Financial crisis, US unconventional monetary policy and international spillovers} według jego autorów jest fakt, iż obniżenie spreadu korporacyjnego (i~w mniejszym stopniu spreadu za czas do wykupu) miało istotny wpływ na warunki finansowe i~aktywność ekonomiczną w~Stanach Zjednoczonych, ale także w~innych analizowanych krajach. Patrząc na amerykańską gospodarkę szoki na spreadach instrumentów dłużnych według autorów analizowanego artykułu mogły wpłynąć statystycznie istotnie na: wzrost PKB realnego na poziomie 0,15-0,2 punktu procentowego, wzrost inflacji o~około 0,1 p.p., wzrost wskaźnika zmienności w~pierwszym miesiącu o~koło 6\% i~powolne wygasanie tego efektu w~dalszej perspektywie czasowej oraz wzrost cen akcji o~około 1\% oraz osłabienie kursu dolara o~0,5\%. Reakcja zmiennej obrazującej wartość udzielonych kredytów sektorowi prywatnemu nie okazała się jednoznaczna i~statystycznie istotna. Chen, Filardo, He oraz Zhu stwierdzili ponadto, iż pierwsze luzowanie ilościowe miało silne cechy antycykliczne i~przyczyniło się do uniknięcie długotrwałej recesji i~deflacji w~analizowanych krajach. Miało ono jednak silniejszy wpływ na kraje rozwijające się niż rozwinięte i~mogło się przyczynić do "przegrzania się" gospodarek Chin i~Brazylii w~latach 2010-2011. Autorzy analizowanego artykułu udowadniają również, iż niekonwencjonalna polityka monetarna Rezerwy Federalnej miała istotny wpływ na wzrost cen na giełdach w~Stanach Zjednoczonych, ale także w~innych częściach świata. Finalnym wnioskiem wypływającym z~omawianego artykułu jest stwierdzenie, iż przeprowadzone badanie dostarcza dowodów na to, że polityka monetarna \acs{FED} po 2008~roku mogła prowadzić do powstania globalnych makroekonomicznych i~finansowych niestabilności. Największą słabością omawianego w~tym fragmencie rozdziału artykułu wydaje się założenie \textit{implicite}, iż niekonwencjonalną politykę monetarną Rezerwy Federalnej reprezentują ściśle zmiany w~spreadach amerykańskich obligacji skarbowych i~korporacyjnych bez  zweryfikowania tego założenia w~pierwszej fazie badania. 

W~artykule \textbf{\textit{Effects of US Quantitative Easing on Emerging Market Economies}} jego autorzy S. Bhattarai (Uniwersytet Teksański w~Austin), A. Chatterjee (Uniwersytet Nowej Południowej Walii) oraz W. Y. Park (Uniwersytet Illinois w~Urbana-Champaign) podejmują próbę zbadania wpływu niekonwencjonalnej polityki pieniężnej Rezerwy Federalnej na gospodarkę Stanów Zjednoczonych oraz na gospodarki krajów rozwijających się: Brazylii, Chile, Kolumbii, Indii, Indonezji, Malezji, Meksyku, Peru, Republiki Południowej Afryki, Południowej Korei, Tajwanu, Tajlandii oraz Turcji. Artykuł ten został opublikowany w~listopadzie 2015~roku, jako dokument roboczy, na stronie internetowej Banku Rezerwy Federalnej w~Dallas \cite{bhattarai36}.

Modelem badawczym wykorzystanym przez autorów omawianego artykułu był bayesowski model wektorowej autoregresji (ang. \textit{Bayesian Vector Autoregression - BVAR})  - został on zaimplementowany w~dwóch krokach. W~pierwszym oszacowany został model \acs{BVAR} z~sześcioma opóźnieniami dla danych z~gospodarki Stanów Zjednoczonych wykorzystujący sześć zmiennych endogenicznych: wartość papierów wartościowych w~bilansie Rezerwy Federalnej, rentowność 10-letnich amerykańskich obligacji skarbowych, indeks S\&P500, nominalny efektywny kurs walutowych, index produkcji przemysłowej oraz deflator \acs{PCE}. Drugim krokiem było wykorzystanie uzyskanych szoków niekonwencjonalnej polityki monetarnej jako zmiennej egzogenicznej w~osobnych modelach \acs{BVAR} dla gospodarek krajów rozwijających. Zmiennymi endogenicznymi w~tych modelach były krajowe odpowiedniki zmiennych wykorzystanych w~modelu dla Stanów Zjednoczonych. Większość zmiennych wykorzystywanych w~badaniu została przekształcona logarytmicznie. Okresem badawczym dla wszystkich analizowanych gospodarek były dane miesięczne za okres: styczeń 2008 - listopad 2014. Wyniki oszacowanych modeli \acs{BVAR} zostały przeanalizowane przez autorów artykułu za pomocą funkcji odpowiedzi na impuls oraz dekompozycji wariancji błędów prognoz.

W~swoim artykule Bhattarai, Chatterjee i~Park dochodzą do wniosku, iż niekonwencjonalna polityka monetarna Rezerwy Federalnej miała statystycznie istotny wpływ na długoterminowe stopy procentowe w~Stanach Zjednoczonych, rynek akcji, poziom produkcji, poziom cen, oczekiwania inflacyjne, rentowność obligacji korporacyjnych i~listów zastawnych,kurs dolara i~poziom cen domów. Oceniają oni, iż jednostkowy, nieoczekiwany szok niekonwencjonalnej polityki pieniężnej na poziomie 40~miliardów dolarów mógł wpłynąć na następujące zachowanie zmiennych, w~analizowanym modelu, w~perspektywie 12~miesięcy: obniżenie rentowności 10-letnich amerykańskich obligacji skarbowych o~10~punktów bazowych, wzrost cen akcji o~50 p.b., wzrost dochodu o~0,4\% oraz wzrost poziomu cen o~0,1\%. Z~kolei patrząc na gospodarki krajów rozwijających się szok ten statystycznie istotnie wpływał na umocnienie się krajowej waluty względem dolara (+25 p.b.), spadek długoterminowych stóp procentowych (-3 p.b.) oraz wzrost cen akcji na krajowej giełdzie (+100 p.b.). 

Ostatnim artykułem badawczym, który zostanie omówiony w~niniejszym rozdziale jest artykuł \textbf{\textit{Measuring the Effects of Unconventional Monetary Policy on Asset Prices}}, którego autorem jest E. T. Swanson z~Uniwersytet Kalifornijskiego. Został on opublikowany w~grudniu 2015~roku na stronach internetowych {National Bureau of Economic Research (NBER)} \cite{swanson37}. Głównym celem autora omawianego artykułu było zbadanie wpływu na amerykańską gospodarkę luzowania ilościowego oraz \textit{forward guidance}, czyli jednych z~najważniejszych wymiarów polityki monetarnej Rezerwy Federalnej stosowanych po 2009~roku.

Metodami badawczymi w~artykule Swansona były: model najmniejszych kwadratów oraz analiza głównych składowych (ang. \textit{Principal Component Analysis - PCA}). Badanie odbyło się w~dwóch etapach. W~pierwszym autor przeanalizował reakcję notowań kontraktów terminowych na komunikaty po posiedzeniach amerykańskich władz monetarnych z~okresu: styczeń 2009 - czerwiec 2015~i~za pomocą analizy głównych składowych wyodrębnił komponent komunikatu związany z~luzowaniem ilościowym oraz komponent związany z~\textit{forward guidance}. Drugim etapem badania było wykorzystanie komponentów uzyskanych z~\acs{PCA} jako zmiennych niezależnych w~modelach najmniejszych kwadratów, gdzie zmiennymi zależnymi były 30-minutowe zmiany następujących zmiennych: rentowności amerykańskich obligacji skarbowych o~rożnych długościach, indeksu S\&P500, kurs USD/EUR, kurs USD/JPY, rentowności oraz spready obligacji korporacyjnych z~ratingiem AAA~i~BBB. Dla tak zbudowanych modeli \acs{MNK} zbadano istotność ich oszacowań po czym przystąpiono do ich interpretacji.

Najważniejszymi obserwacjami wypływającymi z~artykułu \textit{Measuring the Effects of Unconventional Monetary Policy on Asset Prices} są: istotny wpływ ogłoszenia luzowania ilościowego na obniżenie rentowności amerykańskich obligacji skarbowych z~zapadalnością od 2~do 30~lat (od -1 p.b. do -7 p.b. przy szoku równym jednemu odchyleniem standardowemu) oraz na podwyższenie wartości indeksu S\&P500, kursu USD/EUR oraz kursu USD/JPY (wzrost o~około 0,2\%-0,37\%); istotny wpływ ogłoszenia \textit{forward guidance} na podwyższenie rentowności amerykańskich obligacji skarbowych z~zapadalności od 6 miesięcy~do 10~lat (od 1 p.b do 4 p.b.) oraz na obniżenie wartości S\&P500, kursu USD/EUR oraz kursu USD/JPY (spadek w~przedziale 0,19\%-0,25\%). Dodatkowo autor omawianego artykułu wykazał statystycznie istotny wpływ ogłoszenie programu luzowania ilościowego na spadek rentowności obligacji korporacyjnych (-5 p.b.) oraz wzrost ich spreadów (+4 p.b.) przy braku istotnego wpływu \textit{forward guidance}. Największą słabością analizowanego artykułu wydaje się podzielenie komunikatów Rezerwy Federalnej na jedynie dwie kategorie: tych dotyczących luzowania ilościowego oraz tych dotyczących \textit{forward guidance}. Do żadnej z~tych kategorii nie można z~pewnością zaliczyć wszystkich informacji przekazywanych po posiedzeniach amerykańskich władz monetarnych - na przykład komunikatów dotyczących operacji Twist, stąd specyfikacja modelu może wydawać się nieadekwatna.

Przeanalizowanie wybranych artykułów badawczych dotyczących niekonwencjonalnej polityki monetarnej Rezerwy Federalnej pozwala uzyskać ogólny obraz co do celów badawczych, metodyki badań, zbiorów danych oraz najważniejszych wniosków jakie pojawiają się w~literaturze dotyczącej badanego zjawiska. Zaczynając od warstwy metodologicznej: większość autorów w~swoich badaniach posługuje się modelem wektorowej autoregresji lub modelem najmniejszych kwadratów przy wykorzystaniu danych miesięcznych dla zmiennych z~rynków dłużnego, pieniężnego, akcji, walutowego oraz poziomu cen i~produkcji. Prawie wszyscy autorzy dobierali próbki badawcze w~taki sposób, aby analizować tylko obserwacje z~okresu po wybuchu światowego kryzysu gospodarczego, z~tym że długość ich próbek różni się ze względu na moment, w~którym artykuł był przez nich pisany. \hyperlink{tab0}{Tabela 1} przedstawia podsumowanie najważniejszych wniosków wypływających z~przedstawionej w~bieżącym rozdziale analizy literatury badawczej w~kontekście wpływu niekonwencjonalnej polityki monetarnej na wybrane sektory amerykańskiej gospodarki.

\newpage
\begin{landscape} %strony horyzontalnie
\hypertarget{tab0}{}
\begin{table}[!ht]
\rowcolors{2}{lightgray}{white}
\captionsetup{format=hang, position=top}
\caption{Wpływ niekonwencjonalnej polityki monetarnej na wybrane sektory amerykańskiej gospodarki według przeanalizowanej literatury.}
\begin{tabular}{M{5cm}M{3cm}M{2.5cm}M{3cm}M{3cm}M{2.5cm}M{3cm}}
\toprule
\textbf{Tytuł artykułu} & \textbf{Poziom dochodu} & \textbf{Poziom cen} & \textbf{Rynek obligacji} & \textbf{Rynek akcji} & \textbf{Kurs walutowy} & \textbf{VIX} \\
\midrule
\textit{The Financial Market Effects of the Federal Reserve’s Large-Scale Asset Purchases}\cite{gagnon34} & - & - & [{\color{red}$\downarrow$}] Obniżenie spreadu obligacji skarbowych (-0,5~p.p.) & - & - & - \\
\textit{The Effectiveness of Unconventional Monetary Policy at the Zero Lower Bound: A Cross-Country Analysis}\cite{gambacorta35} & [{\color{ForestGreen}$\uparrow$}] Wzrost realnego PKB (+0,05~p.p.) & [{\color{ForestGreen}$\uparrow$}] Wzrost CPI (+0,02~p.p.) & - & - & - & [{\color{red}$\downarrow$}] Krótkotrwały spadek (-1 p.p.) \\
\textit{The Distributive Effects of Conventional and Unconventional Monetary Policies}\cite{davtyan35} & [{\color{ForestGreen}$\uparrow$}] Wzrost realnego PKB (+0,25\%) & [{\color{ForestGreen}$\uparrow$}] Wzrost CPI (+0,15\%) & - & - & - & - \\
\textit{Financial crisis, US unconventional monetary policy and international spillovers}\cite{chen36} & [{\color{ForestGreen}$\uparrow$}] Wzrost realnego PKB (+0,2~p.p) & [{\color{ForestGreen}$\uparrow$}] Wzrost CPI (+0,1~p.p.) & - & [{\color{ForestGreen}$\uparrow$}] Wzrost cen akcji (+1\%) & [{\color{red}$\downarrow$}] Deprecjacja kursu doloara (-0,5\%) & [{\color{ForestGreen}$\uparrow$}] Krótkotrwały wzrost (+3\%) \\
\textit{Effects of US Quantitative Easing on Emerging Market Economies}\cite{bhattarai36} & [{\color{ForestGreen}$\uparrow$}] Wzrost indeksu produkcji przemysłowej (+0,4\%) & [{\color{ForestGreen}$\uparrow$}] Wzrost deflatora PCE (+0,1\%) & [{\color{red}$\downarrow$}] Obniżenie rentowności obligacji skarbowych (-0,1~p.p.) & [{\color{ForestGreen}$\uparrow$}] Wzrost cen akcji (+0,5~p.p.) & - & - \\
\textit{Measuring the Effects of Unconventional Monetary Policy on Asset Prices}\cite{swanson37} & - & - & [{\color{red}$\downarrow$}] Obniżenie rentowności obligacji skarbowych (-0,05~p.p.) & [{\color{ForestGreen}$\uparrow$}] Wzrost cen akcji (+0,2\%) &  [{\color{ForestGreen}$\uparrow$}] Aprecjacja kursu dolara (+0,35\%) & - \\
\bottomrule
\end{tabular}
\begin{flushleft}
\hspace{1cm}\textit{\footnotesize{Źródło: Opracowanie własne.}} \\
\end{flushleft}
\vspace{-0.5cm}
\end{table} 
\end{landscape}

\newpage
Analizując omówione w~bieżącym rozdziale artykuły można odnaleźć kilka wspólnych wniosków. Pierwszym z~nich jest stwierdzenie, iż niekonwencjonalne działania amerykańskich władz monetarnych po wybuchu światowego kryzysu gospodarczego miały statystycznie istotny wpływ na gospodarkę Stanów Zjednoczonych w~szczególności na poziom produkcji, poziom inflacji, rentowność amerykańskich obligacji skarbowych, wartość indeksów na lokalnych giełdach oraz poziom zmienności na rynku akcji. Większość badań wskazuje, iż bezpośrednim wpływem zastosowania niekonwencjonalnej polityki monetarnej w~Stanach Zjednoczonych był wzrost poziomu produkcji oraz poziomu cen. Duża część badań podkreśla również silny wpływ niekonwencjonalnej polityki monetarnej Rezerwy Federalnej na wzrost indeksów giełdowych w~Stanach Zjednoczonych oraz obniżenie rentowności obligacji skarbowych i~korporacyjnych na tamtym rynku, podkreślając szczególnie silny wpływ w~tym zakresie pierwszego programu luzowania ilościowego. Wnioski płynące z~badań nie są jednoznaczne w~stosunku do skuteczności niekonwencjonalnej polityki monetarnej na wpływ na kurs walutowy oraz poziom zmienności. Łatwo można zaobserwować, iż część wniosków wypływających z~przeanalizowanych w~bieżącym rozdziale artykułów stoi w~sprzeczności z~hipotezą badawczą niniejszej pracy, która zakłada, iż niekonwencjonalna polityka monetarna Rezerwy Federalnej stosowana od 2008~roku zamiast pobudzać do wzrostu realnego \acs{PKB} oraz~zbliżać amerykańską gospodarkę do pełnego zatrudnienia przyczyniła się przede wszystkim do wygenerowania ponadprzeciętnych wzrostów cen akcji notowanych na giełdach w~Stanach Zjednoczonych. Dlatego też w~kolejnym rozdziale zaprezentowane zostanie badanie własne mające na celu zweryfikowanie hipotezy badawczej i~odniesienie się do wyników przedstawionych w~literaturze badawczej.
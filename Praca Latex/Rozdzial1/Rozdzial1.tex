\newpage
\chapter*{Rozdział 1 \\ \vspace{1cm} \Large{Podstawy teoretyczne niekonwencjonalnej polityki monetarnej}}
\addcontentsline{toc}{chapter}{1. Podstawy teoretyczne niekonwencjonalnej polityki monetarnej}

\noindent Pierwszy rozdział niniejszej pracy został poświęcony przeglądowi i~usystematyzowania dostępnej wiedzy dotyczącej niekonwencjonalnej polityki monetarnej (ang. \textit{Unconventional Monetary Policy - UMP}),  a~w~szczególności jej najważniejszych narzędzi oraz kanałów transmisji do realnej gospodarki. Jasne i~przejrzyste zdefiniowanie pojęć związanych z~niekonwencjonalną polityką monetarną jest o~tyle istotne, iż polityka ta jest zagadnieniem stosunkowo nowym - do czasu wybuchu światowego kryzysu gospodarczego w~2008 roku była ona rzadko opisywana w~literaturze badawczej. Omawiany rozdział został podzielony na trzy części. Pierwsza z~nich skupia się na wstępnym, ogólnym nakreśleniu pojęcia niekonwencjonalnej polityki pieniężnej. Druga część rozdziału została poświęcona szczegółowemu opisowi i~klasyfikacji instrumentów niekonwencjonalnej polityki pieniężnej wraz z~przedstawieniem przykładów ich zastosowania przez władze monetarne kilku krajów świata. Ostatnia części niniejszego rozdziału prezentuje potencjalne kanały transmisji impulsów monetarnych niekonwencjonalnej polityki pieniężnej do realnej gospodarki. Celem omawianego rozdziału będzie więc zapoznanie czytelnika z~najważniejszymi podstawami teoretycznymi  niekonwencjonalnej polityki monetarnej, tak aby mógł on w~pełni zrozumieć sens oraz specyfikę problemu badawczego poruszanego w~niniejszej pracy. Zrozumienie tych kwestii będzie kluczowe do właściwej analizy treści zamieszczonych w~kolejnych rozdziałach oraz interpretacji wniosków wypływających z~całości przeprowadzonego badania.

\phantomsection	
\hypertarget{podroz11}{}
\section*{\large{1.1. Najważniejsze założenia niekonwencjonalnej polityki monetarnej}}
\addcontentsline{toc}{section}{1.1. Najważniejsze założenia niekonwencjonalnej polityki monetarnej}

Terminem niekonwencjonalna polityka monetarna w~literaturze badawczej określa się zazwyczaj zbiór ponadstandardowych działań banków centralnych w~takich krajach jak Stany Zjednoczone, Wielka Brytania, Japonia, Szwajcaria czy kraje strefy euro w~reakcji na wybuch światowego kryzysu gospodarczego w~2008 roku. Niekonwencjonalność tych działań nie wynika jednak z~faktu, iż zostały one pierwszy raz zastosowane po wybuchu ostatniego kryzysu gospodarczego, gdyż większość z~nich była już wcześniej stosowana przez władze monetarne różnych krajów świata na przestrzeni ostatnich 100~lat (w~zmodyfikowanych wariantach). Niekonwencjonalne w~tym przypadku są skala i~intensywność ich stosowania, zakres oddziaływania na rynki finansowe oraz wpływ na wielkość i~kompozycję bilansu banku centralnego oraz bilansów uczestników rynków finansowych zarówno z~sektora prywatnego, jak i~publicznego. Niekonwencjonalne działania banków centralnych ze statusu ciekawostek orientalnej lub prehistorycznej bankowości centralnej zyskały status głównych instrumentów do walki z~największym kryzysem finansowym od czasu wielkiego kryzysu gospodarczego lat 30.~XX~wieku. Stąd też wzbudzają one ogromne zainteresowanie współczesnych ekonomistów oraz przedostały się do świadomości społecznej szerokiej rzeszy ludności na całym świecie. Brak jest jednak jednoznacznej klasyfikacji i~terminologii dotyczącej niekonwencjonalnej polityki monetarnej - w~wielu pracach badawczych pojawiają się często sprzeczne jej definicje, co utrudnia prowadzenie efektywnej debaty na temat jej pożyteczności, czy też skuteczności. Dlatego też w~tej części rozdziału przedstawiona zostanie jasna klasyfikacja i~terminologia dotycząca badanych niekonwencjonalnej polityki pieniężnej tak, aby maksymalnie ułatwić czytelnikowi zrozumienie istoty badanego w~niniejszej pracy problemu unikając mylących uproszczeń i~generalizacji.

\textbf{Niekonwencjonalna polityka monetarna to zespół nadzwyczajnych działań władz monetarnych oddziałujących na gospodarkę danego kraju za pomocą zmian w~wartości lub kompozycji bilansu banku centralnego. Działania te są najczęściej podejmowane w~sytuacji głębokiego kryzysu gospodarczego lub w~jego następstwie, gdy konwencjonalne narzędzia polityki monetarnej nie przynoszą oczekiwanych efektów. Ich głównym celem jest wywarcie presji na obniżenie stóp procentowych na rynku międzybankowym w~momentach wzmożonej niepewności pomiędzy uczestnikami tego rynku - nadmiernej premii za ryzyko}. Cel ten jest realizowany poprzez obniżanie średnio- i~długoterminowych rynkowych stóp procentowych za pomocą zakupów przez władze monetarne dłużnych papierów wartościowych o~znacznej wartości od banków komercyjnych oraz inwestorów instytucjonalnych. Bezpośrednim skutkiem większości takich operacji jest wzrost bazy monetarnej. Niekonwencjonalna polityka pieniężna jest polityką czysto bilansową oznacza to, iż poprzez zmiany w~kompozycji (strukturze udziału poszczególnych aktywów/pasywów lub średniej zapadalności portfela) i~w~wartości bilansu banku centralnego władze monetarne starają się wpłynąć na kompozycję i~wielkość bilansów członków sektora publicznego i~prywatnego. To w~jaki sposób dany niekonwencjonalny instrument oddziałuje na bilans banku centralnego jest najczęściej spotykanym wyznacznikiem ich podziału.

\phantomsection % do hiperlinków dla sekcji w spisie treści
\hypertarget{podroz11}{}
\section*{\large{1.2. Podstawowe instrumenty niekonwencjonalnej polityki monetarnej}}
\addcontentsline{toc}{section}{1.2. Podstawowe instrumenty niekonwencjonalnej polityki monetarnej}

Podstawowe instrumenty niekonwencjonalnej polityki monetarnej można podzielić na dwie grupy ze względu na sposób oddziaływania na bilans banku centralnego: instrumenty powodujące zmianę wartości bilansu banku centralnego oraz instrumenty niepowodujące takiej zmiany - zmieniające jedynie udział/strukturę poszczególnych aktywów w~bilansie lub średni czas zapadalności jego składników.

\phantomsection % do hiperlinków dla sekcji w spisie treści
\hypertarget{podroz111}{}
\subsection*{\large{1.2.1. Instrumenty powodujące zmianę wartości bilansu banku centralnego}}
\addcontentsline{toc}{subsection}{1.2.1. Instrumenty powodujące zmianę wartości bilansu banku centralnego}

Do grupy niekonwencjonalnych instrumentów polityki pieniężnej powodujących zmianę wartości bilansu banku centralnego należy zaliczyć:

\begin{itemize}
\setlength\itemsep{0.05cm}
\item luzowanie ilościowe (ang. \textit{Quantitative Easing - QE}),
\item luzowanie kredytowe (ang. \textit{Credit Easing - CE}),
\item długoterminowe operacje refinansowe (ang. \textit{Long Term Refinancing Operations - LTRO}),
\item ujemne stopy procentowe (ang. \textit{Negative Interest Rates - NIR}).
\end{itemize}

\textbf{Luzowanie ilościowe} jest niekonwencjonalnym instrumentem polityki pieniężnej stosowanym przez władze monetarne w~formie zakupów od banków komercyjnych długoterminowych krajowych obligacji rządowych lub innych długoterminowych instrumentów finansowych gwarantowanych przez skarb państwa. Transakcje te mają zazwyczaj na celu spłaszczenie krzywej dochodowości dłużnych papierów skarbowych poprzez wpływanie na wzrost ich cen, a~finansowane są ze środków pieniężnych uzyskanych przez powiększenie bilansu banku centralnego prowadzące do wzrostu bazy monetarnej. W~trakcie przeprowadzania tej operacji główny nacisk kładziony jest na monitorowanie wzrostu wartości nadobowiązkowych rezerw trzymanych przez banki komercyjne na kontach w~banku centralnym, bardziej istotne w~tym przypadku są więc pasywa banku centralnego, a~nie skład i~kompozycja jego aktywów. Takie działanie władz monetarnych ma skłonić banki komercyjne do obniżenia kosztów uzyskania kredytów dla gospodarstw domowych i~przedsiębiorstw, zwiększyć krajowy popyt wewnętrzny i~inwestycje, a~w~konsekwencji pobudzić wzrost gospodarczy oraz inflację. Bank Japonii jako pierwszy wprowadził i~zastosował instrument o~nazwie luzowanie ilościowe w~latach 2001-2006, aby zwalczyć deflację i~pobudzić wzrost gospodarczy. W~różnych wariantach i~z~różnymi skutkami instrument ten był wykorzystywany po wybuchu kryzysu finansowego z~2008 roku przez banki centralne Stanów Zjednoczonych, Wielkiej Brytanii, strefy euro oraz Japonii.

Szczególnym przypadkiem zastosowania luzowania ilościowego są działania Szwajcarskiego Banku Narodowego (SBN) na tamtejszym rynku pieniężnym w~sierpniu 2011~roku. Działania te różniły się od klasycznej definicji luzowania ilościowego głównie tym, iż władze monetarne Szwajcarii skupowały od banków komercyjnych przede wszystkim krótkoterminowe papiery wartościowe (bony pieniężne), a~nie długoterminowe obligacje skarbowe - niekonwencjonalna polityka monetarna miała pomóc w~dalszym obniżaniu krótkoterminowych stóp procentowych\footnote{Część badaczy sugeruje, iż pomimo przeprowadzania interwencji jedynie na rynku pieniężnym Szwajcarski Bank Narodowy doprowadził również do obniżenia rentowności długoterminowych obligacji skarbowych poprzez znaczące zwiększenie wartości rezerw nadobowiązkowych\cite{christensen21}}. Wyjątkowa była też skala interwencji w~relatywnie krótkim czasie - w~ciągu trzech tygodni \acs{SBN} wydał na zakup bonów pieniężnych ponad 170~miliardów franków szwajcarskich czyli równowartość 30\% krajowego \acs{PKB}. Żaden bank centralny nie powiększał swojego bilansu w~tamtym czasie w~tak gwałtownym tempie - od 3~sierpnia do 17~sierpnia 2011~roku wartość bilansu Szwajcarskiego Banku Narodowego wzrosła o~ponad 40\% \cite{christensen21}. Bezpośrednim celem tych niekonwencjonalnych działań \acs{SBN} było doprowadzanie do deprecjacji kursu krajowej waluty. Podobnie jak w~przypadku innych zastosowań luzowania ilościowego kładziono główny nacisk na modelowanie wzrostu wartości rezerw nadobowiązkowych banków komercyjnych w~banku centralnym, co miało doprowadzić do silnego osłabienia kursu franka szwajcarskiego (znaczące obniżenie krótkoterminowych krajowych stóp procentowych miało odstraszyć zagranicznych inwestorów), a~w~konsekwencji pobudzić eksport i~zapobiec deflacji.

\textbf{Luzowanie kredytowe} jest niekonwencjonalnym instrumentem polityki pieniężnej stosowanym przez władze monetarne w~formie zakupów od uczestników rynków finansowych nieskarbowych dłużnych papierów wartościowych, przede wszystkim: obligacji przedsiębiorstw, obligacji przedsiębiorstw sponsorowanych przez rząd (ang. \textit{Government-Sponsored Enterprise - GSE}), papierów komercyjnych przedsiębiorstw, listów zastawnych oraz papierów wartościowych opartych na aktywach (ang. \textit{Asset-Backed Securities - ABS}). Celem tej operacji jest zmniejszenie spreadu kredytowego pomiędzy oprocentowaniem obligacji skarbowych, a~oprocentowaniem nieskarbowych papierów dłużnych o~zbliżonym ratingu. Takie działanie ma doprowadzić do obniżenia rentowności kupowanych instrumentów, zachęcić uczestników rynku do dokonywania innych bardziej ryzykownych inwestycji i~w~konsekwencji doprowadzenie do obniżenia kosztów uzyskania kredytu dla gospodarstw domowych i~przedsiębiorstw. Podobnie jak w~przypadku luzowania ilościowego, luzowanie kredytowe prowadzi do wzrostu wartości bilansu banku centralnego, gdyż zakupy nowych nieskarbowych instrumentów dłużnych odbywają się zazwyczaj za pomocą środków pieniężnych specjalnie na ten cel wygenerowanych. 

W~przeciwieństwie do luzowania ilościowego w~luzowaniu kredytowym władze monetarne nie koncentrują się na stronie pasywów bilansu banku centralnego tylko na stronie aktywów - jak ich kompozycja i~średnia zapadalność wpływa na warunki uzyskania kredytu przez gospodarstwa domowe i~przedsiębiorstwa. Luzowanie kredytowe w~sposób oczywisty wpływa na narażenie bilansu banku centralnego na większe ryzyko niż to jest w~przypadku luzowania ilościowego, gdyż nieskarbowe dłużne papiery wartościowe charakteryzują się dużo niższą płynnością i~większym stopniem ryzyka bankructwa niż papiery gwarantowane przez skarb państwa. Dlatego też stosując ten instrument władze monetarne powinny w~sposób bardziej ostrożny i~precyzyjny dobierać kupowane aktywa, tak by nie narazić swojego bilansu na zbyt dużą ekspozycję na ryzyko. Po 2008~roku z~luzowania kredytowego korzystały banki centralne Stanów Zjednoczonych, Wielkiej Brytanii oraz Japonii.

\textbf{Długoterminowe operacje refinansowe} to jedno z~podstawowych narzędzi Europejskiego Banku Centralnego (EBC) stosowane by zarządzać średnioterminową płynnością banków komercyjnych strefy euro poprzez udzielanie pożyczek pieniężnych pod zastaw papierów wartościowych o~określonej jakości. W~swojej podstawowej formie (zapadalność do 6~miesięcy, wysokie wymagania co do zabezpieczenia, ograniczona dostępność środków, nieistotny wpływ na wielkość bilansu banku centralnego), należy zaliczyć to narzędzie do jednej z~lokalnych modyfikacji transakcji depozytowo-kredytowych. Jednak w~grudniu 2011~roku oraz w~lutym 2012~roku długoterminowe operacje refinansowe zostały zawarte z~bankami komercyjnymi strefy euro na zupełnie innych warunkach niż dotychczas sprawiając, iż można je było zacząć zaliczać do niekonwencjonalnych narzędzi polityki pieniężnej. 

Przede wszystkim wydłużono czas na spłatę pożyczki udzielonej przez bank centralny, zazwyczaj omawiane w~poprzednim akapicie transakcje były zawierane na czas jednego miesiąca\cite{ebc22}, w~szczególnych przypadkach było to 6~miesięcy. Jednak transakcje zawierane w~grudniu 2011~i~w~lutym 2012~miały trzyletnią zapadalność po to by zapewnić długookresową wypłacalność zainteresowanym instytucjom. Poza tym rozszerzono paletę możliwych zabezpieczeń pod te transakcje o~papiery wartościowe z~niskim ratingiem oraz wcześniej nie wykorzystywane instrumenty finansowe, po to by banki komercyjne mogły, w~szczególności, pozbyć się ze swoich bilansów obligacji skarbowych krajów strefy euro zagrożonych bankructwem (tzw. kraje \acs{PIIGS}\footnote{Do krajów \acs{PIIGS} zaliczane były następujące europejskie kraje: Portugalia (ang. \textit{\textbf{P}ortugal}), Irlandia (ang. \textit{\textbf{I}reland}), Włochy (ang. \textit{\textbf{I}taly}), Grecja (ang. \textit{\textbf{G}reece}) oraz Hiszpania (ang. \textit{\textbf{S}pain}).}. Dodatkowo każdy bank zainteresowany zawarciem długoterminowej operacji refinansowej mógł pozyskać od EBC dowolną kwotę pieniężną, nie było ograniczeń co do wartości transakcji. W~końcu koszt pozyskania środków w~ramach transakcji z~przełomu 2011~i~2012~roku był niezwykle korzystny. Stąd też długoterminowe operacje refinansowe z~21 grudnia 2011~roku i~29 lutego~2012 roku cieszyły się ogromnym zainteresowaniem skorzystało z~nich ponad 800~banków komercyjnych strefy euro pozyskując łącznie ponad bilion euro. Tak duży popyt na środki pieniężne spowodował proporcjonalny wzrost wartości bilansu Europejskiego Banku Centralnego oraz nieproporcjonalny wzrost ryzyka w~tym bilansie spowodowany głównie przyjmowaniem w~zastaw toksycznych obligacji skarbowych. Zmieniona została więc kompozycja bilansu \acs{EBC} pod kątem udziału poszczególnych aktywów oraz ryzykowności i~średniej zapadalności zarówno aktywów jak i~pasywów. 

W~czerwcu 2014~roku Europejski Bank Centralny ogłosił program będący modyfikacją programu długoterminowych operacji refinansowych nazywając go programem ukierunkowanych długoterminowych operacji refinansowych (ang. \textit{Targeted Long Term Refinancing Operations - TLTRO}). Główną różnicą w~stosunku do poprzedniego programu był fakt, iż wartość środków, które dany bank komercyjny mógł uzyskać od Europejskiego Banku Centralnego w ramach \acs{TLTRO} wynosiła do 7\% wartości pożyczek, które dany bank udzielił gospodarstwom domowym (z~wyłączeniem kredytów hipotecznych) oraz przedsiębiorstwom niefinansowym. Środki pozyskane w~ramach ukierunkowanych długoterminowych operacji refinansowych mogły mieć maksymalnie czteroletnią zapadalność, a~ich oprocentowanie było stałe i~równe głównej stopie refinansowej w~strefie euro w~momencie ich pozyskania powiększonej o~10 punktów bazowych. W~ramach 7~aukcji \acs{TLTRO}, pomiędzy październikiem 2014~a~czerwcem 2016, Europejski Bank Centralny przekazał bankom komercyjnym w~strefie euro około 425~miliardów euro zastępując wygasające na początku 2015~roku operacje \acs{LTRO}. 10~Marca~2016 roku władze monetarne strefy euro ogłosiły kolejny program ukierunkowanych długoterminowych operacji refinansowych (TLTRO2), który różni się w~stosunku do swojego poprzednika przede wszystkim tym, iż wynagradza on banki, które zwiększą wartość udzielonych pożyczek dla przedsiębiorstw niefinansowym i~gospodarstw domowych o~2,5\% w~ciągu dwóch lat \cite{tahiri33}.

\textbf{Ujemne stopy procentowe} dotyczą oprocentowania nadobowiązkowych rezerw oraz krótkoterminowych depozytów banków komercyjnych w~banku centralnym i~do momentu wybuchu kryzysu finansowego w~2008~roku zastosowane zostały tylko raz - w~latach 70. XX wieku przez Szwajcarski Bank Narodowy. Miało to miejsce tuż po upadku systemu walutowego z~Bretton-Woods, w~latach 1972-1978, celem ich zastosowania była obrona kursu franka szwajcarskiego przed zbyt silną aprecjacją, która osłabiała konkurencyjność eksportu Szwajcarii\cite{megg23}. Po 2008~roku narzędzie to w~podobnym celu wykorzystywała również Szwecja (2012), Dania (2012) oraz ponownie Szwajcaria (2014). 

Europejski Bank Centralny ogłosił 5~czerwca~2014~roku wprowadzenie ujemnych stóp procentowych. Intencją władz monetarnych strefy euro było pobudzenie inflacji oraz wzrostu gospodarczego w~unii walutowej. Miało to nastąpić poprzez skłonienie banków komercyjnych głównie z~północy strefy euro do zmniejszania nadobowiązkowych rezerw deponowanych w~\acs{EBC} za pomocą transakcji z~bankami z~południa oraz poprzez spłacanie zawartych na przełomie 2011~i~2012~roku 3-letnich długoterminowych operacji refinansowych, co z~kolei miało wpłynąć na obniżenie kosztów uzyskania kredytu. Ujemne stopy procentowe poprzez koszty odsetek od utrzymywanego w~banku centralnym kapitału mogą wpływać na stopniowe wycofywanie go z~depozytów, co powinno w~dłuższym okresie prowadzić do zmniejszania bilansu banku centralnego i~zmianę jego kompozycji szczególnie po stronie pasywów.

\phantomsection % do hiperlinków dla sekcji w spisie treści
\hypertarget{podroz112}{}
\subsection*{\large{1.2.2. Instrumenty niepowodujące zmiany wartości bilansu banku centralnego}}
\addcontentsline{toc}{subsection}{1.2.2. Instrumenty niepowodujące zmiany wartości bilansu banku centralnego}

Do grupy niekonwencjonalnych instrumentów polityki pieniężnej niepowodujących zmiany wartości bilansu banku centralnego (wpływających jedynie na strukturę poszczególnych aktywów/pasywów lub średnią zapadalność portfela) należy zaliczyć dwa narzędzia wykorzystywane dotąd przez Europejski Bank Centralny (pierwsze z~nich) oraz Rezerwę Federalną (drugie z~nich):

\begin{itemize}
\setlength\itemsep{0.05cm}
\item luzowanie jakościowe (ang. \textit{Qualitative Easing - QLE}),
\item operację Twist (ang. \textit{Operation Twist - OT}).
\end{itemize}

\textbf{Luzowanie jakościowe} jest niekonwencjonalnym instrumentem polityki pieniężnej stosowanym przez władze monetarne w~formie zakupów niskopłynnych aktywów finansowych obarczonych znaczącym ryzykiem przy jednoczesnej sprzedaży wysokopłynnych aktywów obarczonych niższym ryzykiem. Stosowanie tego narzędzia nie powoduje więc zmiany wartości bilansu banku centralnego, gdyż środki pieniężne na nowe zakupy pozyskiwane są ze sprzedaży posiadanych papierów, następuje jednak narastanie ryzyka w~bilansie banku centralnego przeprowadzającego tego rodzaju operacje. Celem przeprowadzania luzowania jakościowego jest najczęściej chęć wpłynięcia na obniżenie rentowności konkretnej grupy dłużnych papierów wartościowych lub przywrócenie odpowiedniej płynności na konkretnym rynku finansowym. Przykładem zastosowania omawianego w~tym akapicie instrumentu są interwencje Europejskiego Banku Centralnego na rynku obligacji skarbowych krajów strefy euro w~latach 2010-2012, kiedy to \acs{EBC} skupował obligacje krajów \acs{PIIGS} (głównie Grecji), w~celu obniżenia rentowności tych papierów, ale również po to aby wykupić toksyczne aktywa z~bilansów instytucji finansowych strefy euro\cite{bagus23}.

\textbf{Operacja Twist\footnote{Operacja Twist nie jest oficjalną nazwą stosowaną przez władze monetarne Stanów Zjednoczonych. W~momencie wprowadzania tego instrumentu w~2011~roku został on nazwany \textit{Maturity Extension Program - MEP}, czyli Program Wydłużenia Zapadalności.}} jest niekonwencjonalnym instrumentem polityki pieniężnej stosowanym przez władze monetarne w~formie sprzedawania przez bank centralny krótkoterminowych obligacji skarbowych (zapadalność do trzech lat) przy jednoczesnym kupowaniu długoterminowych obligacji skarbowych (zapadalność powyżej pięciu lat). W~ten sposób wydłużana jest średnia zapadalność aktywów znajdujących się w~bilansie banku centralnego (bez wpływu na średnią zapadalność pasywów), skróceniu ulega natomiast średnia zapadalność aktywów znajdujących się w~bilansach uczestników rynku finansowego. Głównym celem operacji Twist jest wpłynięcie na spłaszczenie krzywej dochodowości dłużnych papierów rządowych poprzez podniesienie cen długoterminowych obligacji skarbowych i~obniżenie cen ich krótkoterminowych odpowiedników, co podobnie jak w~przypadku luzowania ilościowego i~kredytowego ma wpłynąć na obniżenie kosztów uzyskania kredytów przez gospodarstwa domowe i~przedsiębiorstwa. Najważniejszą zaletą operacji Twist jest brak wpływu na zmianę wartości bilansu banku centralnego oraz umocnienie krajowego kursu walutowego. Największymi ograniczeniami tej operacji są skończony zasób obligacji długoterminowych, które władze monetarne mogą sprzedać w~danym momencie oraz zaburzenie relacji pomiędzy zapadalności aktywów i~pasywów w~bilansie banku centralnego. Operację Twist zastosowano dotychczas dwukrotnie w~Stanach Zjednoczonych w~latach 1961-1965 oraz 2011-2012.

Analiza wniosków wypływających z~zagadnień poruszanych w~niniejszym podrozdziale wskazuje, iż banki centralne posiadają szeroką paletę niekonwencjonalnych instrumentów polityki pieniężnej, dzięki którym są w~stanie reagować na pojawiające się ryzyka w~systemie finansowym. Jednak skuteczność wszelkich działań władz monetarnych szczególnie w~oddziaływaniu na szeroką gospodarkę zależy w~dużej mierze od drożności i~efektywności kanałów transmisji impulsów polityki monetarnej. Dlatego przed przystąpieniem do analizy empirycznej warto rozpoznać i~wyszczególnić najważniejsze kanały transmisji monetarnej, za pomocą których bank centralny może wpływać na badany system finansowy. Temu zagadnieniu poświęcony został kolejny podrozdział.

\phantomsection	% do hiperlinków dla sekcji w spisie treści
\section*{\large{1.3. Kanały transmisji impulsów niekonwencjonalnej polityki monetarnej}}
\addcontentsline{toc}{section}{1.3. Kanały transmisji impulsów niekonwencjonalnej polityki monetarnej}

W~poprzednich dwóch podrozdziałach przedstawiono definicję niekonwencjonalnej polityki monetarnej oraz wskazano główne narzędzia, za pomocą których banki centralne mogą tę politykę realizować. Jednak wspomniane narzędzia nie działają bezpośrednio na system finansowy i~gospodarczy, zazwyczaj służą one jedynie do stymulowania konkretnych procesów, które w~konsekwencji mają wywołać oczekiwany efekt w~realnej gospodarce. Efekt ten jest osiągany zazwyczaj za pomocą tzw. kanałów transmisji polityki monetarnej, czyli ścieżek, dzięki którym impulsy pieniężne wysyłane przez bank centralny docierają do podmiotów gospodarczych takich jak gospodarstwa domowe czy przedsiębiorstwa. Do najważniejszych z~nich w kontekście niekonwencjonalnej polityki monetarnej należą:

\begin{itemize}
\setlength\itemsep{0.05cm}
\item kanał pożyczek bankowych,
\item kanał cen akcji,
\item kanał kursu walutowego,
\item kanał oczekiwań.
\end{itemize}

\noindent Każdy z~wymienionych powyżej kanałów posiada swoją specyfikę - w~odrębny sposób i~z~różnym natężeniem oddziałuje na realną gospodarkę. Dlatego też warto szczegółowo przeanalizować każdy z~nich, aby w~pełni zrozumieć mechanizmy warunkujące ich funkcjonowanie.

\subsubsection*{\normalsize{Kanał pożyczek bankowych}}

Kanał pożyczek bankowych jest związany ze specjalną rolą banków komercyjnych, które zapewniają dostęp do rynku kredytowego podmiotom, które bez pośrednictwa banku nie mogłyby uczestniczyć w~tym rynku ze względu na niski wolumen potrzebnych środków oraz brak szczegółowej informacji o~ich sytuacji finansowej. Władze monetarne mogą starać się stymulować zmiany w~podaży kredytów udzielanych przez banki komercyjne podmiotom prywatnym za pomocą kształtowania wartości rezerw znajdujących się na kontach w~banku centralnym. Robią to w~szczególności stosując niekonwencjonalną politykę monetarną, gdzie wygenerowane nadobowiązkowe rezerwy silnie przewyższają stany w~konwencjonalnych czasach. Im wyższa wartość nadobowiązkowych rezerw tym banki komercyjne mają silniejszą skłonność do udzielania kredytów, aby zaabsorbować w~swoich bilansach dodatkową płynność. Działanie omawianego w~tym paragrafie kanału transmisji monetarnej w kontekście niekonwencjonalnej polityki monetarnej można zobrazować za pomocą poniższego schematu przyczynowo-skutkowych:

\begin{equation}
UMP\uparrow \Longrightarrow R\uparrow \Longrightarrow L\uparrow \Longrightarrow I\uparrow \Longrightarrow Y\uparrow	
\end{equation}
\vspace{-1cm}
\begin{adjustwidth}{2.5em}{0pt}
{\footnotesize Gdzie:  \\
$UMP$ - niekonwencjonalna polityka monetarna, \\
$R$ - rezerwy banków komercyjnych na koncie w~banku centralnym, \\
$L$ - zagregowana wartość udzielonych pożyczek, \\
$I$ - wydatki inwestycyjne (również wydatki konsumpcyjne), \\
$Y$ - zagregowana produkcja.}
\end{adjustwidth}
\vspace{0.3cm}

\noindent Ważną implikacją zaprezentowanego powyżej rozumowania jest fakt, iż niekonwencjonalna polityka pieniężna oddziałująca przez kanał pożyczek bankowych powinna silniej wpływać na wydatki inwestycyjne małych i~średnich firm niż dużych przedsiębiorstw. Dzieje się tak gdyż to właśnie małe i~średnie firmy mają silnie ograniczony dostęp do rynku kredytowego (nie są w~stanie emitować obligacji) i~muszą polegać na kredytach udzielonych im przez banki.

Podstawowym zamysłem władz monetarnych większości krajów świata była naprawa wadliwie działającego po kryzysie kanału pożyczek bankowych za pomocą kilku uzupełniających się mechanizmów. Pierwszym z~nich było przywrócenie płynności na rynku międzybankowym, poprzez większą elastyczność banku centralnego przy zawieraniu operacji repo z~bankami komercyjnymi (m.in. większy wachlarz instrumentów akceptowalnych jako zabezpieczenie, dłuższy okres zawierania transakcji, większa ilość dopuszczonych do operacji podmiotów). Kolejnym etapem naprawy kanału pożyczek bankowych było wygenerowanie na kontach banków komercyjnych w~banku centralnym ogromnych nadmiarowych rezerw poprzez zakup od nich długoterminowych instrumentów dłużnych. Miało to zarówno obniżyć rentowność tych instrumentów (zniechęcając banki do dalszego inwestowania w~nie), jak i~udostępnić bankom środki, które te mogłyby przeznaczyć na udzielanie nowych kredytów. Skuteczność opisanych powyżej działań udrażniających kanał kredytowy jest przedmiotem analizy w~dalszej części niniejszej pracy.

\subsubsection*{\normalsize{Kanał cen akcji}}

Kanał cen akcji można podzielić na dwa mechanizmy: związany z~teorią q~Tobina oraz związany z~efektem majątkowym oddziałującym na konsumpcję. Teoria q~Tobina dostarcza podstaw do analizy w~jaki sposób polityka monetarna może wpływać na gospodarkę poprzez zmiany wycen akcji. Została ona opisana przez Jamesa Tobina w~1969~roku w~artykule \textit{A general equilibrium approach to monetary theory} i~zakładała wyznaczanie współczynnika q~powstałego przez podzielenie wyceny kapitału na giełdzie przez ekonomiczny koszt jego odtworzenia. Współczynnik miał pomóc w~interpretacji zachowań przedsiębiorstw na giełdach - gdy q~jest wyższe od 1, wycena rynkowa firm przewyższa koszt odtworzenia kapitału, co skłania je do emisji nowych akcji i~uzyskania za nie wysokiej ceny. Tak pozyskane środki firmy mogą przeznaczyć na zakupy dóbr inwestycyjnych napędzając tym samym gospodarkę. Współczynnik q~poniżej 1 zniechęca przedsiębiorstwa do pozyskiwania kapitału na giełdzie poprzez emisję akcji, są one bardziej skłonne kupować inne już istniejące firmy - nowe wydatki inwestycyjne nie są generowane.

Niekonwencjonalna polityka monetarna poprzez zwiększenie podaży pieniądza w~gospodarce powinna prowadzić do relatywnego zmniejszenia się zysków uzyskiwanych z~obligacji i~innych instrumentów dłużnych oraz do powstania nadmiarowych środków płynnych w~portfelach podmiotów rynkowych (podaż pieniądza przewyższająca popyt), co może przełożyć się na wzrost inwestycji w~akcje, wzrost ich cen, podwyższenie q~Tobina - w~konsekwencji mobilizując firmy do pozyskiwania kapitału na giełdzie i~reinwestowania go w~rozwój własnych biznesów. Mechanizm ten prezentuje następujący uproszczony schemat:		

\begin{equation}
UMP\uparrow \Longrightarrow P_e\uparrow \Longrightarrow q\uparrow \Longrightarrow I\uparrow \Longrightarrow Y\uparrow	
\end{equation}
\vspace{-1cm}
\begin{adjustwidth}{2.5em}{0pt}
{\footnotesize Gdzie: \\
$P_e$ - ceny akcji, \\ 
$q$ - q~Tobina.}
\end{adjustwidth}
\vspace{0.3cm}

Alternatywny kanał transmisji impulsów niekonwencjonalnej polityki monetarnej za pomocą cen akcji ma miejsce poprzez wpływ rzeczywistego majątku na konsumpcję. Na tym kanale skupił się słynny ekonomista Franco Modigliani konstruując swój model cyklu życia. Zakładał on, iż wydatki konsumpcyjne są determinowane strumieniem zasobów z~całego życia danej jednostki - zasoby te definiował jako kapitał ludzki, kapitał rzeczywisty oraz majątek finansowy. Zakładając więc, że główny komponent majątku finansowego większości konsumentów stanowią akcje, wzrost ich cen powinien spowodować wzrost majątku finansowego danego konsumenta, co przekłada się na wyższą bieżącą wycenę jego strumienia dochodów, która powinna prowadzić do wzrostu bieżącej konsumpcji, a~w~rezultacie do wzrostu zagregowanego krajowego dochodu. Mając w~pamięci wpływ polityki monetarnej na cenę akcji, nakreśloną w~tym paragrafie teorię można przedstawić następująco: 

\begin{equation}
UMP\uparrow \Longrightarrow P_e\uparrow \Longrightarrow W\uparrow \Longrightarrow C\uparrow \Longrightarrow Y\uparrow	
\end{equation}
\vspace{-1cm}
\begin{adjustwidth}{2.5em}{0pt}
{\footnotesize Gdzie: \\
$W$ - majątek,\\
$C$ - konsumpcja.}
\end{adjustwidth}
\vspace{0.3cm}

\noindent Podobne rozumowanie można zastosować do innego głównego składnika majątku konsumentów jakim są nieruchomości, zakładając, iż teoria q~Tobina dobrze odzwierciedla zachowania uczestników rynku nieruchomości.

\subsubsection*{\normalsize{Kanał kursu walutowego}}

Niekonwencjonalna polityka monetarna może oddziaływać na realną gospodarkę przez kanał kursu walutowego wpływając poprzez wzrost podaży pieniądza w~gospodarce \textit{ceteris paribus} na zmniejszenie atrakcyjności krajowych depozytów w~odniesieniu do depozytów zagranicznych, co z~kolei wpływa na osłabienie się krajowej waluty w~stosunku do walut zagranicznych. Zmiana nominalnego kursu walutowego ma bezpośredni wpływ na opłacalność eksportu netto, a~w~konsekwencji kształtuje poziom dochodu danego kraju. Omawiany mechanizm transmisji monetarnej można zapisać za pomocą następującego uproszczonego schematu przyczynowo-skutkowego:

\begin{equation}
UMP\uparrow \Longrightarrow FX_n\downarrow \Longrightarrow E\downarrow \Longrightarrow NX\uparrow \Longrightarrow Y\uparrow	
\end{equation}
\vspace{-1cm}
\begin{adjustwidth}{2.5em}{0pt}
{\footnotesize Gdzie: \\
$FX_n$ - nominalny kurs walutowy,\\
$E$ - nominalne kurs walutowy,\\
$NX$ - eksport netto.}
\end{adjustwidth}
\vspace{0.3cm}

\subsubsection*{\normalsize{Kanał oczekiwań}}

Kanał oczekiwań jest najbardziej bezpośrednim i~jednocześnie najmniej podlegającym kontroli kanałem transmisji impulsów polityki monetarnej. Jest on również kanałem nie oddziałującym poprzez zmianę wielkości bazy monetarnej, a~wykorzystującym politykę informacyjną banku centralnego. Bezpośredniość tego kanału wynika z~faktu, iż za jego pomocą władze monetarne są w stanie na bieżąco kształtować oczekiwania inflacyjne podmiotów gospodarczych wpływając tym samym na ich decyzje inwestycyjne/zakupowe, a~w~konsekwencji na poziom cen. Pomimo tej bezpośredniości, władze monetarne nie są w~stanie dokładnie kontrolować siły zmiany cen jaką wywołają, mogą jedynie starać się umacniać spójność, wiarygodność i~przewidywalność swojej polityki monetarnej, aby móc z~tego kanału w~miarę efektywnie korzystać. Sygnałami przedostającymi się do podmiotów gospodarczych za pomocą kanału oczekiwań są zazwyczaj decyzje banku centralnego co do ustalenia poziomu stóp procentowych, bezpośredniego celu inflacyjnego lub zapowiedzi przyszłych zmian polityki monetarnej. Podmioty gospodarcze na podstawie dostarczonych im informacji kształtują swoje projekcje dotyczące przyszłego poziomu stóp procentowych i~koniunktury gospodarczej, które przekładają na oczekiwaną inflację oraz oczekiwane zwroty z~inwestycji. Mając na uwadze te wskaźniki podejmują one decyzje inwestycyjne/zakupowe, które wpływają bezpośrednio na poziom cen w~gospodarce i~jej zagregowany dochód. 

Najnowszym krokiem banków centralnych w~kierunku skuteczniejszego werbalnego zarządzania oczekiwaniami inflacyjnymi, a~zatem dbania o~stabilność cen, było wprowadzenie instrumentu o~nazwie \textit{forward guidance}, czyli informowania opinii publicznej o~polityce pieniężnej w~okresie dłuższym niż do następnego posiedzenia jednostki decyzyjnej. Komunikaty takie można podzielić na trzy grupy: czasowe (ang. \textit{calendar-based}), jakościowe (ang. \textit{qualitative}) i~progowe (ang. \textit{threshold-based}) w~zależności od przekazanej informacji dotyczącej celu polityki, długości jej trwania lub wartości progowych, które muszą zostać spełnione, żeby implementacja danej polityka została wstrzymana. \textit{Forward guidance} przez niektórych jest zaliczany do niekonwencjonalnych instrumentów polityki pieniężnej, ze względu na jego spopularyzowanie w~czasie ostatniego światowego kryzysu gospodarczego oraz silny wpływ na zachowanie się inwestorów na rynkach finansowych\footnote{W~niniejszej pracy \textit{forward guidance} nie jest uznawane za niekonwencjonalny instrument polityki monetarnej ze względu na brak istotnego wpływu na wielkość i kompozycję bilansu banku centralnego.}. Instrument ten szczególnie często był stosowany przez Rezerwę Federalną w~latach 2008-2016.

W~ramach kanału oczekiwań ogłoszenie stosowania niekonwencjonalnej polityki pieniężnej polegającej na silnym poszerzaniu bazy monetarnej powinno prowadzić do zdecydowanego podwyższenia oczekiwań inflacyjnych powodując tym samym znaczący wzrost poziomu cen. Rozumowanie to można przedstawić za pomocą następującego schematu przyczynowo-skutkowego:

\begin{equation}
UMP\uparrow \Longrightarrow i_F\downarrow \Longrightarrow OI\uparrow \Longrightarrow I\uparrow \Longrightarrow Y\uparrow	
\end{equation}
\vspace{-1cm}
\begin{adjustwidth}{2.5em}{0pt}
{\footnotesize Gdzie: \\ 
$i_F$ - oczekiwany poziom przyszłych stóp procentowych,\\
$OI$ - oczekiwania inflacyjne.}
\end{adjustwidth}
\vspace{0.3cm}

W~bieżącym rozdziale zarysowane zostały podstawy teoretyczne niekonwencjonalnej polityki monetarnej poczynając od jej spójnej definicji poprzez wykorzystywane przez nią narzędzia na głównych kanałach transmisji tej polityki do realnej gospodarki kończąc. Celem niniejszego rozdziału było przedstawienie najważniejszych kwestii związanych z~niekonwencjonalną polityką monetarną, aby umożliwić dogłębną analizę kolejnych części niniejszej pracy. 